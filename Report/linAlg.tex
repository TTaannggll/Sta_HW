\newif\ifvimbug
\vimbugfalse

\ifvimbug
\begin{document}
\fi

\exercise{Linear Algebra Refresher}
 

\begin{questions}

%----------------------------------------------

\begin{question}{Matrix Properties}{5}
A colleague of yours suggests matrix addition and multiplication are similar to scalars, 
thus commutative, distributive and associative properties can be applied.
Is the statement correct? 
Prove it analytically or give counterexamples (for both operations) 
considering three matrices $ A, B, C$ of size $n\times n$.

\begin{answer}\end{answer}
    For the matrix addition:\\
    there is only commutative and commutative, no distributive.
    Both commutative and associative can be applied. To prove it, firstly, we define\\
    \begin{align*}
        A = ( \begin{array}{c c c} 
            a_{11} & \cdots & a_{1n} \\
            \vdots & a_{ij} & \vdots \\ 
            a_{n1} & \cdots & a_{nn} \end{array} )  \quad
        B = ( \begin{array}{c c c} 
            b_{11} & \cdots & b_{1n} \\
            \vdots & b_{ij} & \vdots \\
            b_{n1} & \cdots & b_{nn} \end{array} )  \quad
        C = ( \begin{array}{c c c} 
            c_{11} & \cdots & c_{1n} \\
            \vdots & c_{ij} & \vdots \\
            c_{n1} & \cdots & c_{nn} \end{array} )
    \end{align*}
    
    
    commutative:  $ A + B  =  B + A$, \\ \quad $ (A+B)_{ij}  =  A_{ij} + B_{ij} $, 
    \quad $(B + A)_{ij}  =  b_{ij} + a_{ij} = a_{ij} + b_{ij} = (A+B)_{ij} $ \\
    so $ A + B  =  B + A$   \\ 
    associative:  $(A+B)+C = A +(B+C)$, \\ 
    $((A+B)+C)_{ij} = ( a_{ij} + b_{ij} ) + c_{ij} =  a_{ij} + ( b_{ij}  + c_{ij} ) = (A+(B+C))_{ij} $ \\
    so $ (A+B)+C = A+(B+C) $ \\
    For the matrix multiplication:\\
    the distributive and associative properties can be applied, but commutative not, to prove:\\
    commutative:
    \begin{align*}
        A = ( \begin{array}{c c c} 
            1 & 2 & 3 \\
            4 & 5 & 6 \\ 
            4 & 3 & 2 \end{array} )  \quad
        B = ( \begin{array}{c c c} 
            3 & 2 & 1 \\
            6 & 5 & 4 \\ 
            2 & 3 & 4 \end{array} )  
    \end{align*}
    \begin{align*}
        A * B = ( \begin{array}{c c c} 
            21 & 21 & 21 \\
            54 & 51 & 48 \\ 
            34 & 29 & 24 \end{array} )  \quad
        B * A = ( \begin{array}{c c c} 
            15 & 19 & 23 \\
            42 & 49 & 56 \\ 
            30 & 31 & 32 \end{array} )  
    \end{align*}
    distributive:\\
    $(A+B)*C=A*C+B*C$\\
    $((A+B)*C)_{ij} = \sum_{k=1}^n { (a_{ik}+b_{ik} )*c_{kj} } = \sum_{k=1}^n { a_{ik}*c_{kj} } + \sum_{k=1}^n{b_{ik}*c_{kj} }
    = (A*C + B*C  )_{ij} $\\
    associative:\\
    $(A*B)*C = A*(B*C)$\\
    $(A*B)*C = \sum_{j=1}^n (A*B)_{ij}*C_{jk} = \sum_{j=1}^n \sum_{l=1}^n a_{il}b_{lj}c_{jk} 
    = \sum_{l=1}^n {a_{il}} \sum_{j=1}^n(b_{lj}*c_{jk}) = A*(B*C) $
\end{question}

%----------------------------------------------

\begin{question}{Matrix Inversion}{6}
Given the following matrix 
\begin{equation*}
     A = ( \begin{array}{c c c} 
     1 & 2 & 3 \\
     1 & 2 & 4 \\
     1 & 4 & 5 \end{array} )
\end{equation*}
analytically compute its inverse $ A^{-1}$ and illustrate the steps.

If we change the matrix in
\begin{equation*}
     A = ( \begin{array}{c c c} 
     1 & 2 & 3 \\
     1 & 2 & 4 \\
     1 & 2 & 5 \end{array} )
\end{equation*}
is it still invertible? Why?

\begin{answer}
????????????????????????????????????????????\\
The inverse matrix can be calculate by the equation:
\begin{equation*}
    A^{-1} = \frac{A^{*}}{|A|}, \quad (A^{*})^{T}_{ij} = (-1)^{i+j}*M_{ij}
\end{equation*}
$A^{*}$ is the adjugate matrix, $M_{ij}$ is minor of A, 
i.e. the determinant of the $(n - 1)*(n - 1)$ matrix that results from deleting row i and column j of A.\\ 

It will not be invertible for the new matrix, because the determinante $|A| = 0$, i.e. it does not have the full rank.
And as to the pointed equation $A^{*} / |A|$ to calculate the inverse matrix, it's obvious not invertible.
\end{answer}
\begin{equation*}
    A^{-1} = \frac{1}{} ( \begin{array}{c c c} 
    1 & 2 & 3 \\
    1 & 2 & 4 \\
    1 & 2 & 5 \end{array} )
\end{equation*}

\end{question}
	
%----------------------------------------------

\begin{question}{Matrix Pseudoinverse}{3}
	Write the definition of the right and left Moore-Penrose pseudoinverse of a generic matrix $A \in \R^{n\times m}$.
	
	Given $A \in \R^{2 \times 3}$, which one does exist? Write down the equation for computing it, specifying the dimensionality of the matrices in the intermediate steps.
	
\begin{answer}\end{answer}
\end{question}

%----------------------------------------------

\begin{question}{Eigenvectors \& Eigenvalues}{6}
What are eigenvectors and eigenvalues of a matrix $A$? Briefly explain why they are important in Machine Learning.

\begin{answer}\end{answer}

\end{question}

%----------------------------------------------

\end{questions}
